\documentclass[10pt, twocolumn]{article}
%nobalancelastpage

\usepackage{labreportstyle}
\usepackage{listings}

\lhead{M. Rossetter}
\rhead{Interquark Potential Modelling of Charmonium States}
\title{Interquark Potential Modelling of Charmonium States}
\author{M. Rossetter \\ Physics Problem Solving Computing Project}
\date{\vspace{-0.30cm} \today}

\begin{document}
\twocolumn[
\maketitle
\begin{onecolabstract}
    This is an abstract.
\end{onecolabstract}]
\thispagestyle{plain}

\begin{itemize}
    \item relativistic effects more prominent in charmonium
    \item talk in intro more about calculation of masses and looking for accuracy to experiment
\end{itemize}
\section{Introduction}
This study will use the quarkonium system of the charm quark, known as charmonium, as a means of studying how to model the interquark potential generated by the strong force.
Due to the complex nature of the strong force, it is difficult to study strong force processes using standard methods in many cases. 
The interactions of the strong force at very small distances ($<0.5\,fm$) and large distances ($\sim2\,fm$) are well characterised, but a single description over all distances, and the intermediary distances, has not yet been devised. \\
The strong force generates bound systems of three quarks (baryons) or one quark and an anti-quark (mesons). 
Studying mesons is an integral part of learning about the strong force as it provides a much simpler case than the three-body systems of baryons, and can be used to build up knowledge of the strong force before applying it to more complicated systems. 
Quarkonia, the mesons formed of a quark and its own flavour anti-quark, provide the simplest mesons to study in the case of the heavier quarks, where asymptotic freedom comes into play and the strong coupling constant $\alpha_s$ is small enough that perturbation theory can be used to study systems.
From these requirements, the quarkonia most suited to study are charmonium and bottomonium, as their masses lie within the asymptotic freedom regime and unlike toponium, their lifetimes are long enough that they can feasibly be studied.
For this study, the approximation that relativistic effects are negligible for charmonium, which is confirmed by estimates of quark velocities. 

\section{Non-Relativisic Potential Modelling}
An issue with modelling charmonium is that there is no potential that can be simply derived like in electromagnetism, as there is no uniform theory for the interaction across the range of radius considered.
Theoretical descriptions of quark interactions confine certain aspects of potential models, expecting certain behaviour as $r$ tends either to $0$ or $\infty$.
Any potential models considered then are phenomenological, and will differ in how their creators choose to interpret experiment and describe the potential uniformly based on the behvaiour at small and large distances.
These phenomenological models will never be wholly accurate to the real potential that is at work, but some may describe quarkonia systems to a practical accuracy for some states.\\
An interquark potential is expected to have a Coulomb-like part to prevent the quarks from getting too close and annihilating, as well as a linear growth part which confines the quarks to the system and prevents them from becoming free particles.
These two parts are brought together in the first model which will be used in this study - the Cornell potential,
\begin{equation}
    V_1(r) = -\frac43\frac{\alpha_s}{r} + \beta r,
\end{equation}
where $\frac43$ is derived from colour charge, $\alpha_s$ is the strong coupling constant whose value is dependent on the mass of the quarks being considered, and $\beta$ is a constant to be determined. \\
The Cornell potential balances the short-range and long-range interactions in two different terms, although many other potentials try to combine both these interactions into one term through analysis of experimental data. 
A quasilogarithmic potential,
\begin{align}
    V_2(r) &= cr^d + b,
\end{align}
is one such potential that is derived from the interpretation of colour confinement in quarks and approximate constant that is the splitting of the $2^3S_1$ and $1^3S_1$ states in charmonium and bottomonium. 
$c,d,$ and $b$ are arbitrary constants of the model. \\
Equations (1) and (2) are the potentials used in this study, although there are many other models known for their success, some of which will be discussed in Section \RN{7}.

\section{Finding the Charmonium Wavefunction}
The spin-averaged wavefunction for charmonium will satisfy the 3D Schrodinger equation,
\begin{equation}
    -\frac{\hbar^2}{2\mu}\nabla^2\psi + \left[V(r) - E_{nl}\right]\psi = 0,
\end{equation}
where $\mu = \frac{m_c}{2}$, and $m_c$ is the mass of a charm quark.
Through separation of variables, $\psi = R_{nl}Y^m_l$, the radial wavefunction, $u_{nl} = rR_{nl}$, can be simplified into a set of first order ODEs, 
\begin{align}
    \frac{du_{nl}}{dr} &= v_{nl}, \\
    \frac{dv_{nl}}{dr} &= \frac{l(l+1)}{r^2}u_{nl} - 2\mu\left[E_{nl} - V(r)\right]u_{nl},
\end{align}
These can then be solved for $u_{nl}$ for each energy eigenstate.
Boundary conditions must be set in the limit as r tends to 0.
Ignoring normalisation for now, we require that
\begin{align}
    u_{nl}(0) &= 0, \\
    \frac{du_{nl}(0)}{dr} &= 1.
\end{align}
Once $u_{nl}$ has been solved for, it is normalised such that
\begin{equation}
    \int_0^\infty r^2|R_{nl}|^2dr = \int_0^\infty |u_{nl}|^2dr = 1.
\end{equation}
Experiment measures the mass of charmonium states rather than their energies, so it is easier to use the masses for comparison.
For a spin-averaged quarkonia energy $E_{nl}$ and a quark mass $m_q$, the spin-averaged bound mass,
\begin{equation}
    M_{nl} = E_{nl} + 2m_q.
\end{equation}


\section{Methodology}
A program was written using Python to solve the Schrodinger equation as discussed in Section \RN{3}, using initial estimates for $E_{nl}$, and normalised using Simpson's method.
Equations (4) and (5) were put into scipy.integrate's odeint function to solve for $u_{nl}$. 
The array of values for the radius must be start slightly away from 0 due to numerical innacuracies in the solver.
Due to the nature of the solution, most energies will diverge to $\pm\infty$, and only the correct energy for the particular (n,l) state will yield a solution that abides by the requirements of normalisation.
To find the correct energy, the bisection method was used.
The program was first tested on the simpler case of the Hydrogen atom, before being applied to charmonium.

\subsection{Bisection Method}
The bisection method finds the wavefunction for initial guesses of energies and iterates over the range of these to find the correct value for the energy of the particular spin-averaged eigenstate.
Two initial energies are guessed, defining a range between them where the correct energy is believed to lie.
A third energy is then calculated as $E_2 = \frac12 (E_1+E_3)$.
For each of these energies, the wavefunction's nodes and turning points are counted - for a quarkonium wavefunction $u_{nl}$, there will be $(n-1)$ nodes and $n$ turning points.
If this is not the case for any of the particular energies used, then new energies must be tried. 
If the number of nodes and turning points differs between two of the three energies, then the correct energy will be somewhere between these two. 
These two energies are set as the new $E_1$ and $E_3$, and the process is performed again.
This method is repeated until the energies converge on an energy with the correct nodes and turning points, which will be $E_{nl}$.

\subsection{Hydrogen Wavefunction}
To test the functionality of the program, it was first applied to the case of the Hydrogen wavefunction. 
Using $V_1(r)$, this can be found precisely with a working version of this program, using the following changes from the problem for charmonium:
\begin{equation}
    \frac43 \alpha_s \to \alpha=\frac{1}{137},\; \beta \to 0,\; \mu \to m_e.
\end{equation}
Unlike charmonium, the energy levels of Hydrogen are not dependent on the azimuthal quantum number, $l$, so for its wavefunction $u_{nl}$, it will have $(n-l-1)$ nodes and $(n-l)$ turning points.
The precise solutions for the energy levels of Hydrogen should follow, confirming the program to be ready for charmonium.

\subsection{Calculating Parameters}
The parameters of the two potentials have many values documented in literature that could be used, but these will vary depending on the value of the charm quark mass used; here $m_c = 1.34\,\text{GeV}/c^2$. 
Values from literature could be used, but the accuracy of the calculation of the masses would suffer, if converging solutions could be found at all.\\
The initial method considered was using the bisection method by fixing all but one of the constants from literature and using the known value of the spin-averaged charmonium ground state, $M_{10} = 3.068\,\text{GeV}/c^2$ to vary over the unknown constant instead of energy.
This method can work well for $V_1$ as $\alpha_s$ can easily be fixed from literature for whichever value is chosen for the charm mass. 
Once $\beta$ is found from the bisection method, the potential is fully known and can be used in the bisection method to find energies. \\
However, when considering $V_2$, it is clear that this method will likely be less accurate.
Unlike $V_1$, where $\alpha_s$ is derived from something more than just the model, all three parameters of $V_2$ are arbitrary and will always be specific to each paper recording their values. \\
From considering the limitations with $V_2$, another method for finding the parameters was considered. 
The function applying the bisection method over energy was put into scipy.optimise's minimise package, requiring it to find the optimum potential parameters to minimise the difference between the known ground state energy and the found ground state energy. 

\subsection{Hyperfine splitting of S states}

\section{Results}
\begin{center}
    \begin{tabular}{|c|c|c|c|}
        \hline 
        \rowcolor{lightgray} State & $V_1$ & $V_2$ & Experiment \\
        \hline
        1S & $3.068$ & $3.068$ & $3.068$ \\
        \hline
        1P & $3.522$ & $3.557$ & $3.525$ \\
        \hline
        2S & $3.717$ & $3.745$ & $3.674$ \\
        \hline
        2P & $4.025$ & $4.008$ & $3.933$ \\
        \hline
        3S & $4.197$ & $4.139$ & $4.028$ \\
        \hline
    \end{tabular}
\end{center}
\begin{itemize}
    \item some hyperfine splitting
    \item maybe some more stuff as we go
\end{itemize}
\begin{center}
    \begin{tabular}{|c|c|c|c|}
        \hline
        \rowcolor{lightgray} State & $V_1$ & $V_2$ & Experiment \\
        \hline
        $1^1S_0$ & $2.576$ & $2.630$ & $2.980$ \\
        \hline
        $1^3S_1$ & $3.232$ & $3.214$ & $3.096$ \\
        \hline
        $2^1S_0$ & $3.375$ & $3.502$ & $3.638$ \\
        \hline
        $2^3S_1$ & $3.831$ & $3.826$ & $3.686$ \\
        \hline
        $3^1S_0$ & $3.898$ & $3.965$ & $4.020$ \\
        \hline
        $3^3S_1$ & $4.297$ & $4.197$ & $4.040$ \\
        \hline
    \end{tabular}
\end{center}

\section{Discussion}
\begin{itemize}
    \item higher energy states lose accuracy
    \item hyperfine splitting stuff again
    \item if it works better for charm or bottom?
    \item success of method overall
    \item issues with initial values and guesses of initial energy
    \item validity of potentials
    \item coupling constant varies with energy of state, not constant
    \item some other stuff probably
\end{itemize}

\section{Outlook}
\begin{itemize}
    \item might merge this with discussion
    \item other potentials?
    \item logarithmic one at least
    \item other steps forward, e.g. hyperfine splitting for all states and lifetimes
    \item other methods of solving - more complicated but probably more accurate
\end{itemize}

\section{Conclusion}

\begin{thebibliography}{}
    \bibitem{1} J. Donoghue et al, \textit{Dynamics Of The Standard Model} (Cambridge University Press, Cambridge, 1992), pp. 350-381.
    \bibitem{2} A. Martin, Physics Letters B 100, (1981).
    \bibitem{3} A. Bykov, I. Dremin, and A. Leonidov, Soviet Physics Uspekhi 27, (1984).
    \bibitem{4} E. Eichten et al, Reviews of Modern Physics 80, (2008).
\end{thebibliography}

\end{document}

